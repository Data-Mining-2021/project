%%
%% This is file `sample-sigconf.tex',
%% generated with the docstrip utility.
%%
%% The original source files were:
%%
%% samples.dtx  (with options: `sigconf')
%% 
%% IMPORTANT NOTICE:
%% 
%% For the copyright see the source file.
%% 
%% Any modified versions of this file must be renamed
%% with new filenames distinct from sample-sigconf.tex.
%% 
%% For distribution of the original source see the terms
%% for copying and modification in the file samples.dtx.
%% 
%% This generated file may be distributed as long as the
%% original source files, as listed above, are part of the
%% same distribution. (The sources need not necessarily be
%% in the same archive or directory.)
%%
%% The first command in your LaTeX source must be the \documentclass command.
\documentclass[sigconf]{acmart}

\makeatletter
\newcommand\aepath[1]{%%
	\bgroup
	\ttfamily
	\ae@path#1\relax\@nil
	\egroup}
\def\ae@path#1#2\@nil{%%
	\def\ae@continue{}%%
	\detokenize{#1}\unskip\penalty\z@  
	\ifx\relax#2%%
	\else 
	\def\ae@continue{\ae@path#2\@nil}%%
	\fi
	\ae@continue}
\makeatother

\let\texttt\aepath

\makeatletter
\renewcommand\@formatdoi[1]{\ignorespaces}
\makeatother

%%
%% \BibTeX command to typeset BibTeX logo in the docs
\AtBeginDocument{%
	\providecommand\BibTeX{{%
			\normalfont B\kern-0.5em{\scshape i\kern-0.25em b}\kern-0.8em\TeX}}}

%% Rights management information.  This information is sent to you
%% when you complete the rights form.  These commands have SAMPLE
%% values in them; it is your responsibility as an author to replace
%% the commands and values with those provided to you when you
%% complete the rights form.
\setcopyright{acmcopyright}
\copyrightyear{2018}
\acmYear{2018}
\acmDOI{10.1145/1122445.1122456}

%% These commands are for a PROCEEDINGS abstract or paper.
\acmConference[Woodstock '18]{Woodstock '18: ACM Symposium on Neural
	Gaze Detection}{June 03--05, 2018}{Woodstock, NY}
\acmBooktitle{Woodstock '18: ACM Symposium on Neural Gaze Detection,
	June 03--05, 2018, Woodstock, NY}
\acmPrice{15.00}
\acmISBN{978-1-4503-XXXX-X/18/06}


%%
%% Submission ID.
%% Use this when submitting an article to a sponsored event. You'll
%% receive a unique submission ID from the organizers
%% of the event, and this ID should be used as the parameter to this command.
%%\acmSubmissionID{123-A56-BU3}

%%
%% The majority of ACM publications use numbered citations and
%% references.  The command \citestyle{authoryear} switches to the
%% "author year" style.
%%
%% If you are preparing content for an event
%% sponsored by ACM SIGGRAPH, you must use the "author year" style of
%% citations and references.
%% Uncommenting
%% the next command will enable that style.
%%\citestyle{acmauthoryear}

%%
%% end of the preamble, start of the body of the document source.
\begin{document}
	
	%%
	%% The "title" command has an optional parameter,
	%% allowing the author to define a "short title" to be used in page headers.
	\title{Group 10: COVID-19 Tweets by World Leaders}
	\subtitle{Intermediate Report}
	
	%%
	%% The "author" command and its associated commands are used to define
	%% the authors and their affiliations.
	%% Of note is the shared affiliation of the first two authors, and the
	%% "authornote" and "authornotemark" commands
	%% used to denote shared contribution to the research.
	\author{Anh Tran}
	\author{Gray Buckley}
	\author{Jakob Horvath}
	\author{Tegan Tingley}

	
	%%
	%% Keywords. The author(s) should pick words that accurately describe
	%% the work being presented. Separate the keywords with commas.
	\keywords{datasets, natural language processing, text tagging}
	
	%% A "teaser" image appears between the author and affiliation
	%% information and the body of the document, and typically spans the
	%% page.
	
	%%
	%% This command processes the author and affiliation and title
	%% information and builds the first part of the formatted document.
	\maketitle
	
	\section{Project Choice}
	For the project proposal, we proposed using Twitter as our data source. However, we had three project ideas that we proposed. We have decided to pursue our first project idea - analyzing COVID-19 tweets by world leaders and then comparing the language used in these tweets.
	
	\section{Twitter Handles}
	In order to begin this project, we needed to define a list of world leaders and find their Twitter handles. Due to issues with translating other languages into English and maintaining the meaning of the Tweet, we decided to limit our queries to Twitter accounts that tweeted in English at least most of the time. 
	We made a list of countries around the world and compiled Twitter handles for high-ranking leaders in those countries. For example, for the United States, we decided to query tweets written by:
	\begin{itemize}
		\item President Biden
		\item Former President Trump
		\item Nancy Pelosi, Speaker of the House
		\item Mitch McConnell, Former Senate Majority Leader
		\item Andrew Cuomo, Governor of New York
		\item Ron DeSantis, Governor of Florida
	\end{itemize}
	While Andrew Cuomo and Ron DeSantis do not hold positions in the federal government, they received a lot of media attention during the pandemic. 
	For other countries, we included official government Twitter handles, such as @10DowningStreet the official UK Prime Minister account. 
	After this process was complete we had a list of XX Twitter handles to query. 
	\section{Obtaining Data}
	We initially planned to use Python's TweetPy library, however due to limitations with this library, we instead decided to use the Python  library \textit{snscrape}. This can be used to scrape information from various different social media platforms, including Twitter, Reddit, Facebook, and Instagram. For our purposes, we will only use it to scrape data from Twitter. \cite{snscrape}
	
	\section{Filters}
	While COVID-19 was a major news topic of 2020, this was not the only major news story of the year. Hence, we needed to filter tweets to tweets related to COVID-19. We created a list of words related to the pandemic and then used an OR statement in our query. Here are some of the COVID-19 related words used in our filter. 
	\begin{itemize}
		\item COVID
		\item Mask
		\item Vaccine
		\item Coronavirus
		\item Viral
		\item WHO
		\item CDC
		\item Fauci
	\end{itemize}

	\section{Query}
	Using \textit{snscrape}, we scraped data from Twitter. As mentioned in the previous section, we used a filter to query tweets based on COVID-19 related words. We pulled the date/time of the tweet, tweet ID, user, tweet text, number of likes, number of replies, number of retweets, and the country. We pulled the number of likes, replies, and retweets in order to be able to assess the impact of the tweet. 
	
	A snapshot of some of the data pulled is shown below in ADD FIGURE. 

	\section{Word to Vec}
	After querying all of the tweets by world leaders about COVID-19, our next step was to get the data in a usable form for clustering. We decided to use Word to Vec to help handle this natural language processing task. \cite{wordtovec}
	
	\section{Next Steps}
	We plan to perform clustering on the Twitter data. We plan to use K-Means++, however, we may need to try to use a couple different clustering algorithms if we run into issues. 
	
	The purpose of clustering the data is to try to group words together and then determine which world leaders used those words and how frequently they used the words. This may help us identify relationships between countries, governments, and leadership. 
	
	We would also like to use the number of likes, replies, and retweets to potentially analyze the impact of tweets by world users. We may need to create some sort of baseline by country or leader - i.e. what is a common number of retweets - in order to effectively study this. Tweets with higher than average replies, likes, and retweets will likely have a higher impact than tweets that are below average. 
	

%%
%% The next two lines define the bibliography style to be used, and
%% the bibliography file.

\bibliographystyle{ACM-Reference-Format}
\bibliography{sample-base}
	\begin{enumerate}
		\bibitem{snscrape}
		\texttt{https://github.com/JustAnotherArchivist/snscrape}
		\bibitem{wordtovec}
		\texttt{https://www.tensorflow.org/tutorials/text/word2vec}
	\end{enumerate}
\end{document}
\endinput
%%
%% End of file `sample-sigconf.tex'.
